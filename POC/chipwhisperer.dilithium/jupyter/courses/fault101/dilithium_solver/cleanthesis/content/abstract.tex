% !TEX root = ../my-thesis.tex
%
\pdfbookmark[0]{Abstract / Zusammenfassung}{Abstract / Zusammenfassung}
\addchap*{Abstract}
\label{sec:abstract}

The current asymmetric cryptographic primitives which we use in our daily life are based on the hardness of the factoring and (elliptic curve) discrete logarithm problem. 
Quantum computers may in the near future become powerful enough to solve these problems and break all of our currently used asymmetric cryptographic primitives.
Because of this growing threat the the National Institute of Standards and Technology (NIST) called for proposals of post-quantum secure signature schemes and key encapsulation mechanism (KEM) already back in 2016. This resulted in extensive research on the proposed signatures schemes and KEMs, which provides reasonable assurance that they are very secure in theory.
In practice all algorithms are run on real hardware like microcontrollers and smart cards which are susceptible to physical attacks such as fault attacks. Research on possible fault attacks showed various critical parts of signature algorithms which need to be protected. One particular part in Fiat-Shamir based signature schemes is the sampling of the so-called masking polynomial, which is used to blind the linear linear relationship between the secret key and a publicly known value.
If the sampling of the coefficients of this polynomial is aborted, due to fault, some of the polynomial coefficients will be uninitialized. This will cause information about the secret polynomial to leak which can be used to recover the secret polynomial.
Different countermeasures have been proposed to prevent this attack. One among them is the shuffling countermeasure. The idea behind this countermeasure is to shuffle the order of the coefficients after they have been sampled.

Our results show that this countermeasure is not as effective as believed. Despite this shuffling countermeasure implemented we were still able to recover the secret key.
Our attack is based on Integer Linear Programs (ILPs) to recover the secret polynomial.
We show the effectiveness of our attack by extensive simulations for the signature schemes Dilithium, qTESLA as well as BLISS.
Our results highlights that the shuffling countermeasure alone is not sufficient to prevent fault attacks on the masking polynomial. Additional countermeasure are needed to properly secure upcoming microcontrollers and smart cards running post-quantum secure signature schemes against sophisticated attackers.


%\vspace*{18mm}
\pagebreak
\selectlanguage{german}
{\usekomafont{chapter}Zusammenfassung}
\label{sec:abstract-diff}
\\\phantom{{\usekomafont{chapter}Zusammenfassung}}

Die derzeitigen asymmetrischen kryptographischen Primitive, die wir in unserem täglichen Leben verwenden, basieren auf der Schwere des Faktorisierungsproblems und des diskreten Logarithmus (auf elliptischen Kurven). 
Quantencomputer könnten in naher Zukunft leistungsstark genug sein, um diese Probleme zu lösen und somit alle derzeit verwendeten asymmetrischen kryptografischen Primitive zu brechen.
Aufgrund dieser wachsenden Gefahr hat das National Institute of Standards and Technology (NIST) bereits im Jahr 2016 zur Einreichung von Vorschlägen für post-quantum-sichere Signaturverfahren und Schlüsselkapselungsmechanismen (KEM) aufgerufen.  Daraufhin wurden die vorgeschlagenen Signaturverfahren und KEMs ausgiebig erforscht und in der Theorie als sehr sicher eingestuft.
In der Praxis werden alle Algorithmen auf echter Hardware wie Mikrocontrollern und Chipkarten ausgeführt, die anfällig für physische Angriffe wie Fehlerangriffe sind. Die Forschung zu möglichen Fehlerangriffen zeigte auf, dass verschiedene kritische Bereiche von Signaturalgorithmen geschützt werden müssen.
%Ein besonderer Teil ist die Generierung des Maskierungspolynoms.
Ein besonderer Teil der auf Fiat-Shamir mit Abbruch basierenden Signaturverfahren ist die Generierung des so genannten Maskierungspolynoms, das verwendet wird, um die lineare Beziehung zwischen dem geheimen Schlüssel und einem öffentlich bekannten Wert zu verbergen.
Wenn die Generierung der Koeffizienten dieses Polynoms aufgrund eines Fehlers abgebrochen wird, werden einige der Polynomkoeffizienten nicht initialisiert. Dies führt zum Preisgabe von Informationen über das geheime Polynom, welche zur vollständigen Wiederherstellung des geheimen Polynoms verwendet werden können.
Es wurden verschiedene Gegenmaßnahmen vorgeschlagen, um diesen Angriff zu verhindern. Eine davon ist die Mischen-Gegenmaßnahme. Die Idee hinter dieser Gegenmaßnahme besteht darin, die Reihenfolge der Koeffizienten nach der Abtastung zu mischen.

Unsere Ergebnisse zeigen, dass diese Gegenmaßnahme nicht so effektiv ist wie angenommen. Trotz dieser Gegenmaßnahme war es uns möglich, den geheimen Schlüssel wiederherzustellen.
Unser Angriff basiert auf ganzzahliger linearen Optimieung (Integer Linear Program, ILP), um das geheime Polynom wiederherzustellen.
Wir zeigen die Wirksamkeit unseres Angriffs durch umfangreiche Simulationen für die Signaturverfahren Dilithium, qTESLA und BLISS.
Unsere Resultate zeigen, dass die Mischen-Gegenmaßnahme allein nicht ausreicht, um Fehlerangriffe auf das Maskierungspolynom zu verhindern. Es sind zusätzliche Gegenmaßnahmen erforderlich, um zukünftige Mikrocontroller und Chipkarten, auf denen post-quantum-sichere Signaturverfahren eingesetzt werden, gegen raffinierte Angreifer zu schützen.

\selectlanguage{english}
