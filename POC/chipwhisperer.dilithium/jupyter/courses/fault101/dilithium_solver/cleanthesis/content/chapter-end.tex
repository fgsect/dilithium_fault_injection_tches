% !TEX root = ../my-thesis.tex
%




\chapter{Conclusion, Countermeasures and future work}
Quantum computers may be able to break classic cryptography based on factoring and (elliptic curve) discrete logarithm soon. Thus we need post-quantum secure alternatives now. The NIST standardization process already revealed that Kyber is the recommended post-quantum secure key encapsulation mechanism and Dilithium the recommended post-quantum secure signature scheme to be standardized \cite{niststatus}.
Thus in the near future these schemes will be deployed on real devices, including  smartcards, microcontrollers and other devices susceptible to fault attacks but essential for our daily life. It is of great importance that before these devices are deployed in the real-world it is well known what attacks do exist and how to properly prevent them. While research on PQC fault attacks  has already been conducted \cite{espitau,Groot_Bruinderink_Pessl_2018,sensitivity,faultccm}%
%the effectiveness of the shuffling countermeasure has been overestimated.
, it was still far unclear whether the proposed countermeasures were sufficient.

In this thesis we showed how break one particular countermeasure, the shuffling countermeasure, which was so far assumed to be a sufficient countermeasure against loop-abort faults. 
%In this thesis we showed how to break this countermeasure.
We used multiple faulty signatures to construct a noisy equation system. We were able reduce the noise by filtering certain equations based on basic statistical analysis. Finally we constructed an ILP which was able to remove the last bit of noise and solve the equation system which revealed (part of) the secret key.
We demonstrated the attack effectiveness by attacking lattice-based signature schemes based of the Fiat-Shamir with aborts over rings type. We picked three signature schemes, namely the winner of NIST's PQC standardization process, Dilithium, as well as qTESLA and BLISS. Especially for Dilithium and qTesla's parameter set \texttt{qTESLA-p-I} we showed that the timing of the loop-abort is irrelevant, as long as at least one or two coefficients of the masking polynomial(s) are zero respectively. As such an event can occur with high probability under normal operation conditions for a single signature, it is hard to detect by software using the \say{check for zero} countermeasure. Our results highlight that more offensive research is needed on the proposed countermeasures and more additional countermeasures need to be proposed.
\section{Possible Countermeasures}
Bindel et al. \cite[p.~74]{sensitivity} presented a countermeasure against an attack, which skipped the addition of the entire masking polynomial $y$: Instead of adding the error polynomial on to the $sc$ vector, we add the $sc$ and $y$ into a new variable. This does not directly apply to our attack scenario as we do not cause $y$ to be zero by skipping its addition but instead we skip only part of its addition by skipping iterations in the sampling loop. Still, we can use this idea as a countermeasure against our attack by combining the sampling of $y$ and the addition of $y$ and $sc$ into a single loop, instead of doing both operations separately. Thus if we try to abort the loop which samples $y$, we also abort the addition of $sc$ and $y$. Consequently for all coefficients we fault, we only get uninitialized memory which does not contain any information. While this countermeasure is very efficient it only works for faults which target the execution flow of the program but it does not cover faults which for example target the programs memory.

As a possible countermeasure to protect BLISS, and possible other signature schemes which use the discrete gaussian distribution for their masking vectors, blinded gaussian shuffling can be used. This countermeasure was first introduced by Saarinen \cite[p.~82]{multishuffle} to counteract side-channel attacks against discrete gaussian sampler. The general idea is that then adding two discrete gaussian distributed random variables $X$ and $Y$ with a standard deviation of $\sigma$, centered at zero, $X + Y$ is also follows the discrete gaussian distribution, centered at zero with a standard deviation of $\sqrt{σ^{2} + σ^{2}}$. We can use this fact to construct a $y$-vector with standard deviation $σ$ by adding $k$ discrete gaussian distributed vectors with standard deviation $σ' = \frac{σ}{\sqrt{m}}$. Furthermore we shuffle all vectors before we add them together. This would require an attacker to perform $k$ loop-abort faults and still the expected amount of zero coefficients in the resulting masking vector would be exponentially small in $k$. Latter because a faulted-coefficient of a sampled polynomial might be added with a non-faulted coefficient the other vector due to the shuffling. Additionally this also protects against side-channel attacks. On the other hand this strategy also decreases the performance of the discrete gaussian sampler by $k$-fold. 


\section{Future work}
While we believe that our simulations already stress the need for more countermeasure research it would be interesting to see this attack to be implemented in practice on real hardware to further proof the point.
Additionally it might be possible to improve the performance of the attack by combining the $p$- and $t$-parameter in an optimal fashion. When evaluating \say{surplus of equations} parameter we choose a very conservative threshold.
Especially the qTESLA signature scheme was affected by the very conservative choice of the threshold, as this threshold was also used in the ILP as a constraint for the secret polynomial coefficient variables. A stricter constraint for these variables might improve the ILP performance and thus the attack performance.