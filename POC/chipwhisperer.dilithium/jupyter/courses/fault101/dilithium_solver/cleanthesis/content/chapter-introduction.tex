% !TEX root = ../my-thesis.tex
%

\chapter{Introduction}
The potential advent of large-scale quantum computers threatens to undermine the security of currently used public-key cryptography. Virtually all public-key cryptography that is used right now is based on either the hardness of factoring or discrete logarithm problems. Using Shor's algorithm \cite{shor}, large enough general-purpose quantum computers will be able to break these problems, rendering the respective public-key cryptography insecure.
Because of this the National Institute of Standard and Technology (NIST), in an effort to standardize post-quantum  cryptography, called for proposals for post-quantum secure cryptographic schemes \cite{nistcall}, i.e., schemes that withstand quantum adversaries. The call encompasses signature, encryption and key encapsulations algorithms that can be used in a potential quantum computer era.

The standardization process is currently in the fourth round, with various proposed candidates being eliminated because they were found to be insecure or impractical and some candidates already selected for standardization \cite{niststatus}. The remaining and selected candidates have thus been subject to thorough theoretical analysis.  However, the corresponding implementations require scrutiny on the implementation security, since existing cryptoanalytic efforts focused mainly on the schemes' designs. The foreseeable execution platforms for post-quantum cryptography algorithms include microcontrollers and smart cards, to, e.g., sign banking transactions. Smart cards and microcontrollers are subject to physical attacks, such as fault injection attacks. Especially for public-key cryptography like signature schemes, fault attacks can break the security of a device running a poorly implemented algorithm easily. 
Fault attacks cause the device signing a message to perform a miscalculation and therefore leak secret information. Past research demonstrated fault attacks that revealed  the entire secret with only single or a few faulty signatures \cite{faultseifert}. Designing efficient and effective countermeasures against fault-injection attacks is non-trivial, given the powerful attack primitive that they need to defend against. 

In this thesis, we will investigate the effectiveness of one specific fault-injection countermeasure against so-called loop-abort fault attacks.
%We will focus on lattice-based signature schemes, because the by the NIST to be standardized and recommended signature scheme for PQC is Dilithium \cite{niststatus}, a lattice based signature scheme.
We will focus in lattice based signature schemes because two out of the three signature schemes the NIST will standardize are based in lattices. Additionally the currently recommended PQC signature scheme is Dilithium, a lattice based signature scheme.

Prior work by Espitau et al. \cite{espitau} has shown that lattice-based signature schemes based on Fiat-Shamir with aborts (over ideal lattices) leak secret information if a masking polynomial is not properly sampled. Fault injections can cause such an improper sampling and thus can be used to reveal the secret key. To mitigate this attack, a proposed countermeasure is to randomize the order the sampling of the masking polynomial, i.e. shuffling the order of its coefficient \cite[p.~155]{espitau}. A second countermeasure was to check whether the upper few coefficients are zero and abort if too many are zero  \cite[p.~155]{espitau}. 

\section{Our Contribution}
This thesis shows that the shuffling countermeasure is not as effective as believed. We formulate Integer Linear Programs (ILPs) that can recover the secret key effectively from faulted signatures, even if the shuffling countermeasure is present. The described attack affects the implementation security of signature schemes like BLISS, qTESLA as well as the signature scheme Dilithium \cite{bliss,qtesla,dilithium}.
Dilithium is the winner of the NIST Post-Quantum-Cryptography standardization process \cite{niststatus}.  It can thus be assumed that it will be implemented a lot in the near future.
We will validate the effectiveness of our attack by extensive simulations for all the aforementioned PQC signature schemes.

Furthermore our results will show that we can even prevent the \say{checking for zero} countermeasure for Dilithium and qTESLA as we only require 1 or 2 zeroed coefficients per signature. For a single signature this can happen naturally with a high probability and thus can not be detected.

Our results will highlight the need for further research into securing the implementations of lattice-based signature schemes against fault attacks.